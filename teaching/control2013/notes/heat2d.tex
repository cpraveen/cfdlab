\documentclass[12pt]{article}
\usepackage[width=16cm]{geometry}                % See geometry.pdf to learn the layout options. There are lots.
\geometry{a4paper}                   % ... or a4paper or a5paper or ... 
%\geometry{landscape}                % Activate for for rotated page geometry
%\usepackage[parfill]{parskip}    % Activate to begin paragraphs with an empty line rather than an indent
\usepackage{graphicx}
\usepackage{amssymb}
\usepackage{amsmath}
\usepackage{aliases}
\usepackage{color}
\usepackage{url}

\usepackage{listings}
\usepackage{cancel}
\usepackage{textcomp}

\lstset{
   language=matlab,
   keywordstyle=\bfseries\ttfamily\color[rgb]{0,0,1},
   identifierstyle=\ttfamily,
   commentstyle=\color[rgb]{0.133,0.545,0.133},
   stringstyle=\ttfamily\color[rgb]{0.627,0.126,0.941},
   showstringspaces=false,
   basicstyle=\small,
   numberstyle=\footnotesize,
   numbers=none,
   stepnumber=1,
   numbersep=10pt,
   tabsize=2,
   breaklines=true,
   prebreak = \raisebox{0ex}[0ex][0ex]{\ensuremath{\hookleftarrow}},
   breakatwhitespace=false,
   aboveskip={0.1\baselineskip},
    columns=fixed,
    upquote=true,
    extendedchars=true,
% frame=single,
    backgroundcolor=\color[rgb]{0.9,0.9,0.9}
}

\title{Two dimensional heat equation}
\author{Deep Ray, Ritesh Kumar, Praveen. C, Mythily Ramaswamy, J.-P. Raymond}
\date{}

\begin{document}



\maketitle

\begin{center}
IFCAM Summer School on Numerics and Control of PDE\\
22 July - 2 August 2013\\
IISc, Bangalore\\
\url{http://praveen.cfdlab.net/teaching/control2013}
\end{center}


\section{The PDE model}
Let $z = z(x,y,t)$ denote the temperature. 
The shifted 2-D heat equation is given by
\[
z_t = \mu \Delta z + \omega z, \quad (x,y) \in \Omega = (0,1) \times (0,1), \qquad t \in  [0,T]
\]
with boundary conditions, see figure~(\ref{fig:prob})
\[
z(x,0,t) = z(x,1,t) = 0, \quad z(1,y,t) = u(y,t), \qquad \df{z}{x}(0,y,t) = 0
\]
and initial condition
\[
z(x,y,0) = z_0(x,y)
\]
Here $\omega \ge 0$ and $\mu > 0$. Let us denote the Dirichlet part of the boundary by $\Gamma_D$
\[
\Gamma_D = \{ y=0\} \cup \{ y=1\} \cup \{ x=1 \}
\]
the Neumann part as
\[
\Gamma_N = \{ x=0 \}
\]
and the part on which the control is applied as
\[
\Gamma_c = \{ x=1 \}
\]
\begin{figure}
\begin{center}
\includegraphics[width=0.5\textwidth]{heat2d_prob}
\caption{Problem definition}
\label{fig:prob}
\end{center}
\end{figure}
%------------------------------------------------------------------------------

\subsection{Observations}
We will measure an average value of the temperature on strips along the left vertical boundary
\[
I_i = [a_i, b_i], \qquad i=1,2,3
\]
as shown in figure~(\ref{fig:obs}). Thus the observations are
\begin{equation}
y_i(t) = \frac{1}{b_i - a_i} \int_{a_i}^{b_i} z(0,y,t) \ud y, \qquad i=1,2,3
\end{equation}

\begin{figure}
\begin{center}
\includegraphics[width=0.5\textwidth]{heat2d_obs}
\caption{Observations}
\label{fig:obs}
\end{center}
\end{figure}
\noindent
{\bf Note}: Do not confuse the observation $y_i$ with the spatial coordinate $y$.
%------------------------------------------------------------------------------
\subsection{Weak formulation}
We assume $z_0 \in L^2(\Omega)$. We wish to find $z \in L^2(0,T;H^1(\Omega))$ such that
\[
 \dd{}{t}\int_\Omega z(t) \phi = - \mu \int_\Omega \nabla z \cdot \nabla \phi +  \omega \int_\Omega z \phi, \quad \forall \phi \in H^1_{\Gamma_D}(\Omega)
\]
\[
z(x,0,t) = z(x,1,t) = 0, \quad z(1,y,t) = u(y,t)
\]
\[
 (z(0),\phi)_{L^2} = (z_0 ,\phi)_{L^2}
\]
%------------------------------------------------------------------------------

\section{FEM approximation}
\begin{figure}
\begin{center}
\includegraphics[width=0.5\textwidth]{heat2d_mesh}
\caption{Example of a finite element mesh}
\label{fig:mesh}
\end{center}
\end{figure}
Consider a division of $\Omega$ into disjoint triangles as shown in figure~(\ref{fig:mesh}). We will assume that the vertices of the mesh are numbered in some manner. Let us define the following sets of vertices
\begin{eqnarray*}
N_c &=& \mbox{vertices on $\Gamma_c$} \\
N_d &=& \mbox{vertices on } \{y=0\} \cup \{y=1\} \\
N_f &=& \mbox{remaining vertices (unknown degrees of freedom)}
\end{eqnarray*}
For each vertex $i$, define the piecewise affine functions $\phi_i(x,y)$, see figure~(\ref{fig:basis}), with the property that
\[
\phi_i(x_j, y_j) = \delta_{ij}
\]
\begin{figure}
\begin{center}
\includegraphics[width=0.5\textwidth]{tria_basis}
\caption{Piecewise affine basis functions}
\label{fig:basis}
\end{center}
\end{figure}
Then the finite element solution is of the form
\[
z(x,y,t) = \sum_{j \in N_f} z_j(t) \phi_j(x,y) + \sum_{j \in N_c} u_j(t) \phi_j(x,y), \qquad z(x_i,y_i,t) = z_i(t), \quad \forall i \in N_f
\]
This function already satisfies the Dirichlet boundary conditions
\[
z(x,0,t) = z(x,1,t) = 0, \qquad z(1,y,t) = \sum_{j \in N_c} u_j(t) \phi_j(1,y)
\]
We will take the control to be of the form
\[
u(y,t) = v(t) \sin(\pi y) \qquad \Longrightarrow \qquad u_j(t) = v(t) \sin(\pi y_j)
\]
This is a one dimensional control which is compatible with the other boundary control
\[
u(0,t) = u(1,t) = 0
\]
Then the finite element solution is of the form
\[
z(x,y,t) = \sum_{j \in N_f} z_j(t) \phi_j(x,y) + v(t) \sum_{j \in N_c} \sin(\pi y_j) \phi_j(x,y)
\]
The approximate weak formulation is given as
\[
 \dd{}{t} \int_\Omega z(t) \phi_i = - \mu \int_\Omega \nabla z \cdot \nabla \phi_i +  \omega \int_\Omega z \phi_i, \quad \forall i \in N_f
\]
i.e.,
\begin{eqnarray*}
&& \sum_{j \in N_f} \dd{z_j}{t} \int_\Omega \phi_j \phi_i + \dd{v}{t} \sum_{j \in N_c} \sin(\pi y_j) \int_\Omega \phi_j \phi_i \\
&=& -\mu \sum_{j \in N_f} z_j \int_\Omega \nabla \phi_j \cdot \nabla \phi_i - \mu v \sum_{j \in N_c} \sin(\pi y_j) \int_\Omega \nabla \phi_j \cdot \nabla \phi_i \\
&& + \omega \sum_{j \in N_f} z_j \int_\Omega \phi_j \phi_i + \omega v \sum_{j \in N_c} \sin(\pi y_j) \int_\Omega \phi_j \phi_i, \qquad \forall i \in N_f
\end{eqnarray*}
In order to simplify the presentation we will ignore the term containing $\dd{v}{t}$. This term vanishes if we use the trapezoidal rule for integration. Then we can write the FEM formulation as
\begin{eqnarray*}
&& \sum_{j \in N_f} \dd{z_j}{t} \int_\Omega \phi_j \phi_i \\
&=&  \sum_{j \in N_f} z_j \left[ -\mu\int_\Omega \nabla \phi_j \cdot \nabla \phi_i + \omega \int_\Omega \phi_j \phi_i \right] + \\
&& v \sum_{j \in N_c} \left[ - \mu  \sin(\pi y_j) \int_\Omega \nabla \phi_j \cdot \nabla \phi_i + \omega\sin(\pi y_j) \int_\Omega \phi_j \phi_i \right] \qquad \forall i \in N_f
\end{eqnarray*}
This can be written as a system of ordinary differential equations
\[
 \M \dd{\z}{t} = \A \z + \B v
\]
where $\z$ denotes all the unknown degrees of freedom corresponding to $N_f$.
%-----------------------------------------------------------------
\subsection{Grid information}
The grid information consists of following files
\begin{itemize}
\item {\tt coordinates.dat}: contains $x,y$ coordinates of all the vertices

\item {\tt elements3.dat}: contains verices forming each triangle

\item {\tt dirichlet.dat}: contains boundary edges on the dirichlet boundary

\item {\tt neumann.dat}: contains boundary edges on the neumann boundary

\end{itemize}
An example of this type of data is given in figures~(\ref{fig:fem501}),~(\ref{fig:fem502}). In each file, the first column is the serial number.
\begin{figure}
\begin{center}
\includegraphics[width=0.8\textwidth]{fem50_1}
\caption{Example of a mesh}
\label{fig:fem501}
\end{center}
\end{figure}

\begin{figure}
\begin{center}
\includegraphics[width=0.5\textwidth]{fem50_2}
\includegraphics[width=0.6\textwidth]{fem50_4}
\caption{Structure of mesh files}
\label{fig:fem502}
\end{center}
\end{figure}

\begin{center}
{\bf Note: In our current programs we use a mesh consisting of only triangles. \\
Also, the serial numbers are not stored in the files. }
\end{center}

The domain $\Omega$ is the unit square. The program {\tt square.m} generates the mesh and creates the above four files. You have to specify the number of vertices on the side of the square. E.g., to have 11 points on each side, you do the following in matlab
\begin{lstlisting}
>> square(11)
\end{lstlisting}
When you run the above program, you can see a picture of the mesh. Examine the four files created by this program. For our actual computations, we will use 101 points on each side of the square. In this case, there are 5 edges which exactly cover each of the three observation zones.
%-----------------------------------------------------------------

\subsection{Finite element assembly}
The finite element basis functions have compact support. Hence we can compute the integrals 
\[
\int_\Omega \phi_i \phi_j \qquad \mbox{and} \qquad \int_\Omega \nabla \phi_i \cdot \nabla \phi_j
\]
by adding the contributions from a small number of triangles. For example, the elements of the mass matrix can be computed as
\[
\int_\Omega \phi_i \phi_j = \sum_{K \ : \ i, j \in K} \int_K \phi_i \phi_j
\]
and similarly the stiffness matrix is computed as
\[
\int_\Omega \nabla \phi_i \cdot \nabla \phi_j = \sum_{K \ : \ i, j \in K} \int_K \nabla \phi_i \cdot \nabla \phi_j
\]
The integrals on each triangle $K$ will be evaluated exactly. For a triangle $K$ with vertices labelled 1, 2, 3 as in figure~(\ref{fig:tria}), the local mass and stiffness matrices are given by
\[
M^K = \begin{bmatrix}
\int_K \phi_1 \phi_1 & \int_K \phi_1 \phi_2 & \int_\Omega \phi_1 \phi_3 \\
\int_K \phi_2 \phi_1 & \int_K \phi_2 \phi_2 & \int_\Omega \phi_2 \phi_3 \\
\int_K \phi_3 \phi_1 & \int_K \phi_3 \phi_2 & \int_\Omega \phi_3 \phi_3
\end{bmatrix} = \frac{1}{24} \mbox{det}\begin{bmatrix}
x_2 - x_1 & x_3 - x_1 \\
y_2 - y_1 & y_3 - y_1
\end{bmatrix} \begin{bmatrix}
2 & 1 & 1 \\
1 & 2 & 1 \\
1 & 1 & 2
\end{bmatrix}
\]
\begin{figure}
\begin{center}
\includegraphics[width=0.3\textwidth]{tria}
\caption{Triangle}
\label{fig:tria}
\end{center}
\end{figure}
\[
A^K = \half \mbox{det}\begin{bmatrix}
1 & 1 & 1 \\
x_1 & x_2 & x_3 \\
y_1 & y_2 & y_3
\end{bmatrix} G G^\top \qquad\textrm{where}\qquad G = \begin{bmatrix}
1 &  1 & 1\\
x_1 & x_2 & x_3 \\
y_1 & y_2 & y_3
\end{bmatrix}^{-1} \begin{bmatrix}
0 & 0 \\
1 & 0 \\
0 & 1
\end{bmatrix}
\]
The numerical process can be summarized by the following algorithm:

\vspace{2mm}
\noindent
Set $\M=0$, $\A=0$.\\
For each triangle $K$ in the mesh
\begin{itemize}
\item Compute $M^K$ and $A^K$
\item Add the contributions from $M^K$ into $\M$ and from $A^K$ into $\A$
\end{itemize}
For the $j'$th triangle, the mapping between the local index to global index is given below
\begin{center}
\begin{tabular}{|c|c|}
\hline
Local index & Global index \\
\hline
1 & {\tt elements3(j,1)} \\
\hline
2 & {\tt elements3(j,2)} \\
\hline
3 & {\tt elements3(j,3)} \\
\hline
\end{tabular}
\end{center}
For $j=1,2,\ldots,{\tt Ntri}$\\
\indent For $k=1,2,3$ \\
\indent \indent For $l=1,2,3$
\begin{eqnarray*}
\A( elements3(j,k), elements3(j,l) ) &+=& A^{K_j}(k,l) \\
\M( elements3(j,k), elements3(j,l) ) &+=& M^{K_j}(k,l) \\
\end{eqnarray*}
More details on the assembly process can be found in this paper

\vspace{2mm}\noindent
Jochen Alberty, Carsten Carstensen and Stefan A. Funken: ``Remarks around 50 lines of Matlab: short finite element implementation", Numerical Algorithms 20 (1999) 117�-137\\
\url{http://math.tifrbng.res.in/~praveen/notes/control2013/acf.pdf}
%-----------------------------------------------------------------
\section{Parameters and initial condition}
Let us take
\[
\mu = \frac{1}{50}, \qquad \omega = 0.4
\]
Then the heat equation has one unstable eigenvalue given by
\[
\lambda = -\frac{\pi^2}{40} + \omega = 0.015325988997
\]
The corresponding eigenfunction is
\[
\phi(x,y) = \cos(\pi x/2) \sin(\pi y)
\]
\begin{figure}
\begin{center}
\includegraphics[width=0.9\textwidth]{heat2d_initcond}
\caption{Unstable eigenfunction}
\label{fig:initcond}
\end{center}
\end{figure}
which is shown in figure~(\ref{fig:initcond}). We will use the unstable eigenfunction as the initial condition for the heat equation. This initial condition satisfies the following boundary conditions
\[
\phi(x,0) = \phi(x,1) = \phi(1,y) = 0, \qquad \phi_x(0,y) = 0
\]
%-----------------------------------------------------------------
\section{Excercises I}
The solution of heat equation is implemented in {\tt heat2d.m}. For the case without control, we use zero boundary conditions on $x=1$ which corresponds to setting $v=0$ or equivalently we can set $\K=0$ since $v=-\K\z$.

\begin{enumerate}

\item Generate the mesh by running the {\tt square} program and use 41 or 81 or 101 points on each side of the square.

\item Study the program {\tt get\_matrices.m} which performs the matrix assembly.

\item The value of $\omega$ is set in the file {\tt parameters.m}. Set $\omega=0$ and calculate five eigenvalues with largest real part using {\tt eigs} function. Is there an  unstable eigenvalue~? Use the program {\tt check\_eig.m}.

\item Study the program {\tt heat2d.m} which solves the heat equation.

\item With $\omega=0$, run the program. 
\begin{lstlisting}
>> heat2d(0) % argument 0 = no control => v=0 or K=0
\end{lstlisting}
Observe that the energy is decreasing with time. The energy is plotted with linear scale for time and log scale for energy. The straight line behaviour indicates that the energy is decaying exponentially with time.

\item Set $\omega=0.4$ and calculate five eigenvalues with largest real part using {\tt eigs} function. Is there an unstable eigenvalue~? Compare it with exact unstable eigenvalue given above.

\item With $\omega=0.4$, run the program. Observe that the energy is increasing exponentially with time.

\end{enumerate}

\pagebreak


%-----------------------------------------------------------------
\section{Feedback control}
The feedback control is determined by solving the minimal norm control problem using the matrices $\M$, $\A$ and $\B$. Since the dimensions of these matrices are very large, we determine the feedback matrix $\K$ by considering only the unstable components of the system as we did in the case of 1-D heat equation. This is implemented in the program {\tt heat2d.m}.
%-----------------------------------------------------------------
\section{Excercises II}
\begin{enumerate}

\item Set $\omega=0.4$. Use the program {\tt check\_eig.m} to compute eigenvalues. There should be one unstable eigenvalue.

\item Set $\omega=0.4$. Use Hautus criterion to check that $(\M,\A,\B)$ is stabilizable.

\item For $\omega = 0.4$ run {\tt heat2d.m}.
\begin{lstlisting}
>> heat2d(1)
\end{lstlisting}
After $\K$ is computed inside {\tt heat2d.m}, check that $(\M,\A-\B\K)$ is stable by computing it eigenvalue (only a few eigenvalues with largest real part using {\tt eigs}). Does the solution become stable with this feedback ?

\item Save the matrix $\K$ obtained above. Set $\omega=0$ and load $\K$ into {\tt heat2d.m} instead of evaluating it. Run it and observe the decay rate of the solution. Is it better than the case without control ?

\item Try to solve the Riccati equation for the full system and check how much time it takes. Use $\omega=0.4$
\begin{lstlisting}
>> square(41)
>> globals;
>> parameters;
>> read_data();
>> [M,A,B,Q,H] = get_matrices();
>> R = 1;
>> [P,E,K] = care(full(A),full(B),full(Q),full(R),[],full(M));
\end{lstlisting}
Use this feedback operator $\K$ and solve the heat equation.

\end{enumerate}

%-----------------------------------------------------------------

\section{Estimation and control}
We now consider the case where we have partial observation as discussed in the previous sections. Thus we need to solve the estimation problem. This requires the evauation of the filtering gain matrix $\bL$. Once again, the large dimensions of the system matrices forces us to evaluate $\bL$ using only the unstable components.

%-----------------------------------------------------------------
\subsection{Computing the observation}
\begin{figure}
\begin{center}
\includegraphics[width=0.8\textwidth]{heat2d_obs_mesh}
\caption{Triangular mesh with 100 edges on each side. The figure shows a zoomed view near the left boundary. Notice that the first observation zone  $\{x=0, 0.2 \le y \le 0.25\}$ is exactly covered by five boundary edges.}
\label{fig:heatobsmesh}
\end{center}
\end{figure}
We will assume that the intervals $[a_i, b_i]$ on which the observation is computed is exactly covered by the edges of the finite element mesh, see figure~(\ref{fig:heatobsmesh}). Let $E_i$ denote the edges on $[a_i,b_i]$. Then
\begin{eqnarray*}
y_i &=& \frac{1}{b_i-a_i} \int_{a_i}^{b_i} z(0,y,t) \ud y, \qquad i=1,2,3 \\
&=& \frac{1}{b_i-a_i} \sum_{e \in E_i} \int_e z(0,y,t) \ud y \\
&=& \frac{1}{b_i-a_i} \sum_{e \in E_i} \half (z_{e_1} + z_{e_2}) |e|
\end{eqnarray*}
The set of observations can be written as
\[
\y = \bH \z
\]
The observation zones are defined by the following parameters
\begin{center}
\begin{tabular}{|c|c|c|c|}
\hline
Value/$i$ & 1 & 2 & 3 \\
\hline
$a_i$ & 0.20 & 0.50 & 0.80 \\
\hline
$b_i$ & 0.25 & 0.55 & 0.85 \\
\hline
\end{tabular}
\end{center}


%-----------------------------------------------------------------
\subsection{Excercises III}
The control and estimation problem is implemented in {\tt heat2d\_est.m}.

\begin{enumerate}

\item For $\omega = 0.4$ run {\tt heat2d\_est.m}; compare the true solution and estimated solution by computing the error norm between the two solutions.

\item Use $\omega=0.4$; we now want to achieve better stabilization, so we add a further shift of 0.5 in which case there are three unstable eigenvalues
\begin{lstlisting}
   alpha=0.5;
   opts.p=50; % number of Lanczos vectors
   [V,D] = eigs(A+alpha*M,M,10,'lr',opts);
   ee=diag(D)
   nu = 3;
   V = V(:,1:nu);
   D = D(1:nu,1:nu);
\end{lstlisting}
Note that we add the shift {\tt alpha} only to compute the feedback gain.

\end{enumerate}
\end{document}  
