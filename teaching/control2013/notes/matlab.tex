% $Header: /cvsroot/latex-beamer/latex-beamer/solutions/conference-talks/conference-ornate-20min.en.tex,v 1.6 2004/10/07 20:53:08 tantau Exp $

\documentclass[11pt,xcolor=svgnames,onlymath]{beamer}
%\documentclass[11pt,xcolor=svgnames,onlymath,gray]{beamer}

\mode<presentation>
{
  %\usetheme{Madrid}
  \usetheme{Boadilla}
  % or ...

  \setbeamercovered{transparent}
  \setbeamertemplate{navigation symbols}{}
  % suppress slide number for continued slides
  \setbeamertemplate{frametitle continuation}[from second][]
  \setbeamertemplate{footline}[page number]{}
  % or whatever (possibly just delete it)
}

\usepackage{graphicx}
\usepackage{aliases}
%\usepackage{colortbl}
\usepackage{listings}
\usepackage{textcomp}

\usepackage[english]{babel}
% or whatever

%\usepackage[latin1]{inputenc}
% or whatever

%\usepackage{times}
%\usepackage[T1]{fontenc}
% Or whatever. Note that the encoding and the font should match. If T1
% does not look nice, try deleting the line with the fontenc.

\usefonttheme{serif}
\setbeamertemplate{itemize item}[circle]
\setbeamertemplate{itemize subitem}[triangle]
\setbeamertemplate{enumerate item}[circle]
\setbeamertemplate{enumerate subitem}[square]

% equations will appear in this colour
%\everydisplay{\color{magenta}}

%\everydisplay{\color{blue}}


\title[Lecture 1] % (optional, use only with long paper titles)
{Introduction to Matlab}

%\subtitle{Lecture 1}

\author[Praveen. C] % (optional, use only with lots of authors)
{Praveen. C\\
{\tt praveen@math.tifrbng.res.in}}
% - Give the names in the same order as the appear in the paper.
% - Use the \inst{?} command only if the authors have different
%   affiliation.

\institute[TIFR-CAM] % (optional, but mostly needed)
{
   \includegraphics[height=1.0cm]{tifr.png}\\
   Tata Institute of Fundamental Research\\
   Center for Applicable Mathematics\\
   Bangalore 560065\\
\url{http://math.tifrbng.res.in/~praveen}
}
% - Use the \inst command only if there are several affiliations.
% - Keep it simple, no one is interested in your street address.

\date[IFCAM-2013] % (optional, should be abbreviation of conference name)
{IFCAM Summer School on Numerics and Control of PDE\\
22 July - 2 August, 2013\\
IISc, Bangalore}
% - Either use conference name or its abbreviation.
% - Not really informative to the audience, more for people (including
%   yourself) who are reading the slides online

%\subject{Computational Fluid Dynamics}
% This is only inserted into the PDF information catalog. Can be left
% out. 

% If you have a file called "university-logo-filename.xxx", where xxx
% is a graphic format that can be processed by latex or pdflatex,
% resp., then you can add a logo as follows:

%\pgfdeclareimage[height=0.5cm]{university-logo}{inria}
%\logo{\pgfuseimage{university-logo}}

% Delete this, if you do not want the table of contents to pop up at
% the beginning of each subsection:
%\AtBeginSubsection[]
%{
%  \begin{frame}<beamer>
%    \frametitle{Outline}
%    \tableofcontents[currentsection,currentsubsection]
%  \end{frame}
%}

% If you wish to uncover everything in a step-wise fashion, uncomment
% the following command: 

%\beamerdefaultoverlayspecification{<+->}

\begin{document}

\lstset{
   language=matlab,
   keywordstyle=\bfseries\ttfamily\color[rgb]{0,0,1},
   identifierstyle=\ttfamily,
   commentstyle=\color[rgb]{0.133,0.545,0.133},
   stringstyle=\ttfamily\color[rgb]{0.627,0.126,0.941},
   showstringspaces=false,
   basicstyle=\small,
   numberstyle=\footnotesize,
   numbers=left,
   stepnumber=1,
   numbersep=10pt,
   tabsize=2,
   breaklines=true,
   prebreak = \raisebox{0ex}[0ex][0ex]{\ensuremath{\hookleftarrow}},
   breakatwhitespace=false,
   aboveskip={0.1\baselineskip},
    columns=fixed,
    upquote=true,
    extendedchars=true,
% frame=single,
    backgroundcolor=\color[rgb]{0.9,0.9,0.9}
}

\begin{frame}
  \titlepage
\end{frame}

%\begin{frame}
% \frametitle{Outline}
%\tableofcontents
% You might wish to add the option [pausesections]
%\end{frame}
%#############################################################################
%-------------------------------------------------------------------------------------
\begin{frame}[allowframebreaks,fragile]
%\frametitle{Introduction to Matlab}
In the following slides, the symbol
\begin{lstlisting}
>>
\end{lstlisting}
denotes the matlab command prompt.

\vspace{2mm}

{\bf Variables}: Come into existence when you assign a value
\begin{lstlisting}
>> x=1
\end{lstlisting}
Variable names are case sensitive: {\tt x} and {\tt X} are different. To prevent the value from being printed to screen, end the line with a colon
\begin{lstlisting}
>> x=1;
\end{lstlisting}
You can now use the variable {\tt x} in other statements
\begin{lstlisting}
>> y=sin(x)
\end{lstlisting}

\vspace{2mm}

A {\bf row vector}
\begin{lstlisting}
>> x = [1,2,3,4]
>> y=sin(x)
\end{lstlisting}
Note that Matlab computed {\tt sin} on every element of the vector {\tt x}

\pagebreak

A {\bf column vector}
\begin{lstlisting}
>> x = [1; 2; 3; 4]
>> y = sin(x)
\end{lstlisting}
Output {\tt y} inherits dimensions of input {\tt x}

\vspace{2mm}

{\bf Matrix}
\begin{lstlisting}
>> x = [1, 2, 3, 4; 5, 6, 7, 8]
>> y=sin(x)
\end{lstlisting}

\vspace{2mm}

{\bf Line continuation}
\begin{lstlisting}
>> x = [1, 2, 3, 4; ...
        5, 6, 7, 8]
>> y=sin(x)
\end{lstlisting}

\vspace{2mm}

{\bf Adding vectors}
\begin{lstlisting}
>> x = [1, 2, 3, 4]
>> y = [5, 6, 7, 8]
>> z = x + y
\end{lstlisting}
{\tt x} and {\tt y} must have same dimensions. The following is wrong

\begin{lstlisting}
>> x = [1, 2, 3, 4]
>> y = [5; 6; 7; 8]
>> z = x + y
\end{lstlisting}

\vspace{2mm}

{\bf To find dimensions}
\begin{lstlisting}
>> size(x)
>> size(y)
\end{lstlisting}

\vspace{2mm}

{\bf Transpose} a vector or matrix
\begin{lstlisting}
>> z = x + y'
>> size(y')
\end{lstlisting}

\vspace{2mm}

Find all variables
\begin{lstlisting}
>> who
\end{lstlisting}

\vspace{2mm}

Deleting all existing variables
\begin{lstlisting}
>> clear all
>> who
\end{lstlisting}

\pagebreak

{\bf Matrix-vector multiplication}
\begin{lstlisting}
>> x = [1; 2]
>> A = [1, 2; 3, 4]
>> y = A*x
\end{lstlisting}

\vspace{2mm}

{\bf Matrix-matrix} operations
\begin{lstlisting}
>> B = [5, 6; 7, 8]
>> C = A + B
>> D = A*B
\end{lstlisting}

\vspace{2mm}

{\bf Elementwise operation}
\[
z = x y
\]
\begin{lstlisting}
>> x = [1, 2, 3, 4]
>> y = [5, 6, 7, 8]
>> z = x .* y  % x and y must be of same size
\end{lstlisting}
One can also use a {\tt for} loop, but this will be slow in matlab
\begin{lstlisting}
for j=1:4
   z(j) = x(j) * y(j)
end
\end{lstlisting}

A more complicated example
\[
z = \frac{x^2 \sin(y)}{\cos(x+y)}
\]

\begin{lstlisting}
>> z = x.^2 .* sin(y) ./ cos(x+y)
\end{lstlisting}

Multiply matrices element-wise
\begin{lstlisting}
>> E = A .* B   % A and B must have same size
\end{lstlisting}


{\bf Zero vector/matrix}
\begin{lstlisting}
>> x = zeros(4,1)
>> A = zeros(3,3)
\end{lstlisting}

{\bf Ones vector/matrix}
\begin{lstlisting}
>> x = ones(4,1)
>> A = ones(3,3)
\end{lstlisting}

{\bf Identity matrix}
\begin{lstlisting}
>> A = eye(4)
\end{lstlisting}

\pagebreak

{\bf Random vector/matrix}
\begin{lstlisting}
>> x = rand(1,3) % uniform random variables in [0,1]
>> A = rand(3,2)
\end{lstlisting}

Standard normal random variables: zero mean, unit variance
\begin{lstlisting}
>> x = randn(3,1) % x is in (-inf,+inf)
\end{lstlisting}
Generate values from a normal distribution with mean = 1 and standard deviation = 2
\begin{lstlisting}
r = 1 + 2 * randn(5,1);
\end{lstlisting}

{\bf Documentation}
\begin{lstlisting}
>> help rand
>> help randn
\end{lstlisting}

\end{frame}
%-------------------------------------------------------------------------------------
\begin{frame}[allowframebreaks,fragile]
\frametitle{Plotting}
Making a uniform grid
\begin{lstlisting}
>> x = linspace(0, 2*pi, 10)
>> y = sin(x)
\end{lstlisting}

Plot a line graph
\begin{lstlisting}
>> plot(x, y, '-')
\end{lstlisting}

Plot a symbol graph
\begin{lstlisting}
>> plot(x, y, 'o')
\end{lstlisting}

Plot a line and symbol graph
\begin{lstlisting}
>> plot(x, y, 'o-')
\end{lstlisting}

\newpage 

Multiple graphs
\begin{lstlisting}
>> x = linspace(0, 2*pi, 100);
>> y = sin(x);
>> z = cos(x);
>> plot(x, y, 'b-', x, z, 'r--')
>> xlabel('x')
>> ylabel('y,z')
>> legend('x versus y', 'x versus z')
>> title('x versus y and z')
\end{lstlisting}

\newpage

Subplots
\begin{lstlisting}
>> x = linspace(0, 2*pi, 100);
>> y = sin(x);
>> z = cos(x);
>> subplot(1,2,1)
>> plot(x, y, 'b-')
>> xlabel('x')
>> ylabel('y')
>> subplot(1,2,2)
>> plot(x, z, 'r--')
>> xlabel('x')
>> ylabel('z')
\end{lstlisting}

For more, use {\tt help}
\begin{lstlisting}
>> help plot
\end{lstlisting}

\end{frame}
%-------------------------------------------------------------------------------------
\begin{frame}[allowframebreaks,fragile]
\frametitle{Logarithmic plots}
Suppose we want to plot
\[
y = \exp(-10 x), \qquad x \in [0,1]
\]
Then $y$ varies between $10^{-4}$ to $1$. A normal plot does not clearly show the function, as seen in the left figure. 
\begin{center}
\includegraphics[width=0.5\textwidth]{logplot}
\end{center}
\begin{lstlisting}
>> x = linspace(0,1,100);
>> y = exp(-10*x);
>> figure(1)
>> plot(x,y)
\end{lstlisting}
We can use logarithmic scale for the $y$ axis
\begin{lstlisting}
>> figure(2)
>> semilogy(x,y)
\end{lstlisting}
Now the variation of $y$ is clearly seen. Study the matlab file {\tt logplot.m} included in the {\tt matlab} directory.

\vspace{2mm}

Also check out these other functions for logarithmic plots\\
{\tt semilogx, loglog}
\end{frame}
%-------------------------------------------------------------------------------------
\begin{frame}[allowframebreaks,fragile]
\frametitle{Sparse matrices}
Suppose the matrix $A$ has mostly zero entries
\[
A = \begin{bmatrix}
0 & 1 & 0 \\
0 & 0 & 2 \\
3 & 0 & 0
\end{bmatrix} \qquad \mbox{We will store only the non-zero entries.}
\]
Create a sparse matrix
\begin{lstlisting}
>> A = sparse(3,3)
\end{lstlisting}

At this stage $A$ is empty (zero matrix). Fill in non-zero entries
\begin{lstlisting}
>> A(1,2) = 1;
>> A(2,3) = 2;
>> A(3,1) = 3;
\end{lstlisting}

To get normal matrix
\begin{lstlisting}
>> B = full(A)
\end{lstlisting}

To convert normal matrix to sparse matrix
\begin{lstlisting}
>> C = sparse(B)
\end{lstlisting}

Sparse diagonal matrix
\[
A = \mbox{diag}[1, -2, 1] = \begin{bmatrix}
-2 & 1 & 0 & 0 & 0 & 0 \\
1  & -2 & 1 & 0 & 0 & 0 \\
0  & 1 & -2 & 1 & 0 & 0 \\
0  & 0 & 1 & -2 & 1 & 0 \\
0  & 0 & 0 & 1 & -2 & 1 \\
0  & 0 & 0 & 0 & 1 & -2

\end{bmatrix} \in \re^{n \times n}
\]
\begin{lstlisting}
>> n = 10;
>> e = ones(n,1);
>> A = spdiags([e, -2*e, e], -1:1, n, n);
\end{lstlisting}

Sparse identity matrix
\begin{lstlisting}
>> A = speye(5)
\end{lstlisting}

\end{frame}
%-------------------------------------------------------------------------------------
\begin{frame}[allowframebreaks,fragile]
\frametitle{Eigenvalues and eigenvectors: $Ax=\lambda x$, $Ax = \lambda Bx$}
%\[
%A x = \lambda x
%\]
\begin{lstlisting}
>> A = randn(100,100);
>> lambda = eig(A);
>> plot(real(lambda), imag(lambda), 'o')
\end{lstlisting}

To get eigenvectors also
\begin{lstlisting}
>> A = randn(5)
>> [V,D] = eig(A)
\end{lstlisting}
Columns of {\tt V} contain eigenvectors, 
\[
V = [e_1, e_2, \ldots, e_n] \in \re^{n \times n}, \qquad e_j \in \re^n
\]
{\tt D} is diagonal matrix with eigenvalues on the diagonal
\[
D = \mbox{diag}[\lambda_1, \lambda_2, \ldots, \lambda_n]
\]
\[
Ae_j = \lambda_j e_j \qquad \Longrightarrow \qquad A V = V D
\]

\pagebreak

{\bf Generalized eigenvalues/vectors}
\[
Ax = \lambda Bx
\]
\begin{lstlisting}
>> A = rand(10,10);
>> B = rand(10,10);
>> lambda = eig(A,B);
>> [V,D] = eig(A,B);
\end{lstlisting}

\vspace{2mm}

{\bf Sparse matrices}\\
For large, sparse matrices, we may want to find only few eigenvalues, e.g., those with largest magnitude.
\begin{lstlisting}
>> A = rand(10,10);
>> lambda = eigs(A,2)
\end{lstlisting}
To get eigenvectors and eigenvalues
\begin{lstlisting}
>> [V,D] = eigs(A,2)
\end{lstlisting}
Similarly, to get generalized eigenvectors/values
\begin{lstlisting}
>> A = rand(10,10);
>> B = rand(10,10);
>> lambda = eigs(A,B,2)
>> [V,D] = eigs(A,B,2)
\end{lstlisting}

If matrix is {\bf non-symmetric}, then we may want to compute eigenvalues with {\bf largest real part}
\begin{lstlisting}
>> lambda = eigs(A,B,2,'LR')
>> [V,D] = eigs(A,B,2,'LR')
\end{lstlisting}
Other options available are
\begin{lstlisting}
'SR', 'LI', 'SI'
\end{lstlisting}

\end{frame}
%-------------------------------------------------------------------------------------
\begin{frame}[allowframebreaks,fragile]
\frametitle{Numerical example: {\tt eigtest.m}}
Compute eigenvalues and eigenfunctions
\[
-u''(x) = \lambda u(x), \qquad x \in (0,1)
\]
\[
u(0) = u(1) = 0
\]
Exact eigenvalues and eigenfunctions
\[
u_n(x) = \sin(n\pi x), \qquad \lambda_n = \pi^2 n^2, \qquad n=1,2,\ldots
\]
Use finite difference method: form a grid
\[
0 = x_0 < x_1 < x_2 < \ldots < x_{N+1} = 1, \qquad x_j - x_{j-1} = h = \frac{1}{N+1}
\]
\[
- \frac{u_{j-1} - 2 u_j + u_{j+1}}{h^2} = \lambda u_j, \qquad j=1,2,\ldots,N
\]
\[
u_0 = u_{N+1} = 0
\]
Define
\[
U = [u_1, u_2, \ldots, u_N]^\top, \qquad A = \mbox{diag}[-1, 2, -1] \in \re^{N \times N}
\]
then the finite difference approximation is
\[
AU = \lambda U
\]

{\bf Excercises}

\begin{enumerate}

\item Inside matlab, change directory to the directory {\tt matlab}
\begin{lstlisting}
>> pwd % This shows your current working directory
>> ls  % This shows contents of directory
\end{lstlisting}
You should be able to see the {\tt eigtest.m} file in this directory.

\item Study the program {\tt eigtest.m}

\item Run {\tt eigtest.m}
\begin{lstlisting}
>> eigtest
\end{lstlisting}

\item Compare numerical and exact eigenvalues/eigenfunctions\\
(Eigenfunctions are exact at the grid points. Can you explain why ?)

\item Make a copy of the file {\tt eigtest.m} as {\tt eigtest2.m} \\
In {\tt eigtest2.m}, replace the function {\tt eig} with {\tt eigs} and compute the 5 smallest eigenvalues. When passing matrix {\tt A} to {\tt eigs} function, pass it as a sparse matrix.

\end{enumerate}

\end{frame}
%-------------------------------------------------------------------------------------
\begin{frame}[allowframebreaks,fragile]
\frametitle{Solving system of ODE using {\tt ode15s}}
\[
y \in \re^n, \qquad \dd{y}{t} = \mbox{fun}(t, y, \clr{red}{a, b, c, \ldots}), \qquad {\tt T0} \le t \le {\tt TFINAL}
\]
\[
y({\tt T0}) = y0
\]
Write a matlab program {\tt fun.m} which computes right hand side
\begin{lstlisting}
function f = fun(t, y, a, b, c, ...)
\end{lstlisting}
\begin{center}
\begin{tabular}{|c|l|}
\hline
{\tt tspan} & {\tt [T0, TFINAL]} or {\tt [T0, T1, ..., TFINAL]} \\
& or {\tt T0:dT:TFINAL} \\
\hline
{\tt y0} & Initial condition $y({\tt T0})$ \\
\hline
{\tt options} &  {\tt options = odeset('RelTol',1e-8,'AbsTol',1e-8);} \\
\hline
\end{tabular}
\end{center}

Solve ode
\begin{lstlisting}
[t, Y] = ode15s(@fun, tspan, y0, options, a, b, c, ...)
\end{lstlisting}

\begin{center}
{\tt Y(:,i)} = $i$'th component of solution at different times specified in {\tt tspan}
\end{center}

\end{frame}
%-------------------------------------------------------------------------------------
\begin{frame}[allowframebreaks,fragile]
\frametitle{Numerical example: {\tt odetest.m}}
This program solves the inverted pendulum problem which we will study in next lecture. We will solve the following non-linear ODE
{\footnotesize
\[
\begin{bmatrix}
\dz_1 \\ \dz_2 \\ \dz_3 \\ \dz_4 \end{bmatrix} = \begin{bmatrix}
z_2 \\
\frac{1}{D} [ m l \cos z_3 (c z_4 - mgl\sin z_3) + (I+ml^2)(- k z_2 + ml      z_4^2 \sin      z_3)] \\
z_4 \\
\frac{1}{D}[ (M+m)(-c z_4 + mgl\sin z_3) - ml\cos z_3 (- k z_2 + ml z_4^2     \sin z_3)]
\end{bmatrix}
\]
}
where
\[
D = (M+m)(I+ml^2) - m^2 l^2 \cos^2 z_3
\]
The values of various parameters are set in file {\tt parameters.m}

{\bf Excercises}

\begin{enumerate}

\item Study the programs: {\tt fbo.m, odetest.m}\\
{\tt fbo.m} implements the right hand side function of the ODE\\
{\tt odetest.m} is the driver program which solves the ODE and plots the solution.

\item Run {\tt odetest.m}; you will obtain solution as shown in figure below
\begin{center}
\includegraphics[width=0.6\textwidth]{odetest1}
\end{center}

\item Implement a program to solve the following problem
{\footnotesize
\[
\begin{bmatrix}
\dz_1 \\ \dz_2 \\ \dz_3 \\ \dz_4 \end{bmatrix} = \begin{bmatrix}
z_2 \\
\frac{1}{D} [ m l \cos z_3 (c z_4 - mgl\sin z_3) + (I+ml^2)(\clr{red}{F} - k z_2 + ml      z_4^2 \sin      z_3)] \\
z_4 \\
\frac{1}{D}[ (M+m)(-c z_4 + mgl\sin z_3) - ml\cos z_3 (\clr{red}{F} - k z_2 + ml z_4^2     \sin z_3)]
\end{bmatrix}
\]
}
where
\[
\clr{red}{F} = \alpha u - \beta z_2
\]
\[
u = -Kz, \qquad K = \begin{bmatrix}
-10 & -16.1615 & -71.8081 & -15.2885 \end{bmatrix}
\]
The value of $\alpha$, $\beta$ are set in {\tt parameters.m} file.
\begin{itemize}

\item Copy {\tt fbo.m} as {\tt fbf.m}, e.g. in Unix/Linux\\
%\begin{lstlisting}
\begin{center}
{\tt cp fbo.m fbf.m}
\end{center}
%\end{lstlisting}

\item You have to pass $\alpha$, $\beta$ in the arguments to {\tt fbf} function.

\item Modify {\tt fbf.m} to include the force $F$

\item Copy {\tt odetest.m} as {\tt odetest2.m}

\item Modify {\tt odetest2.m} to now use {\tt fbf} instead of {\tt fbo} and make sure to pass $\alpha$, $\beta$

\item Run {\tt odetest2.m}; you should obtain solution as shown in figure below
\begin{center}
\includegraphics[width=0.6\textwidth]{odetest2}
\end{center}

\end{itemize}

\end{enumerate}
\end{frame}
%-------------------------------------------------------------------------------------
\begin{frame}[fragile]
\frametitle{Some checks}
We will need some functions from the Control System toolbox. Check that you have this toolbox by typing following command
\begin{lstlisting}
>> help lqr
\end{lstlisting}
If you get the message
\begin{lstlisting}
 lqr not found
\end{lstlisting}
then you do not have this toolbox. Talk to one of us.

\end{frame}
%-------------------------------------------------------------------------------------
\end{document}
